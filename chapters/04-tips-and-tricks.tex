\chapter{Tips \& Tricks}

\section{Quellen}
Lade für alle Quellen, die du findest, immer auch das PDF herunter. So kannst du die Dateien in NotebookLM\footnote{\url{https://notebooklm.google/}} hochladen. Es spart erstaunlich viel Zeit, mit den Quellen „sprechen“ zu können und zum Beispiel schnell herauszufinden, wo man bestimmte Informationen gefunden hat.

\section{Nützliche Software}
\subsection{JabRef}
TODO: Link zu Youtube Video einfügen nachdem Siamak das JabRef Tutorial Video hochgeladen hat. 

\subsection{Grafiken}
Falls du matplotlib benutzen willst, findest du in \texttt{/software/matplotlib} ein Skript, das Konstanten definiert, die genau auf die Seitenbreite und Schriftgrößen dieses Dokuments abgestimmt sind. Dieses Dokument hat eine Textbreite von \the\linewidth, und das Skript definiert die entsprechende Breite in Inches. \Cref{fig:test_figure} zeigt ein Beispiel.

\label{subsec:grafiken}
  \begin{figure}[htpb]                                       
      \centering
      \includesvg{figures/test_figure.svg}
      \caption{Beispiel Grafik, mit änlicher Schriftgröße zu dem Text.}
      \label{fig:test_figure}                                     
  \end{figure}        




