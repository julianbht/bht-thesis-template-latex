\chapter{Hinweise}


\section{Häufige Fehler}
In diesem Abschnitt werden besonders schwerwiegende Fehler, die vermieden werden sollten, hervorgehoben. 

\subsection{Keine Trennung zwischen Konzeption und Implementation}

Ein typischer Fehler beim Erstellen des schriftlichen Teils ist eine zu geringe Trennung zwischen Konzeption und Implementierung. In der Regel benötigt es in der Konzeption UML Diagramme. Die Konzeption soll grundsätzlich framework- und sprachunabhängig formuliert werden. Das heißt: Die beschriebenen Strukturen, Abläufe und Überlegungen müssen so allgemein sein,  dass eine Umsetzung in unterschiedlichen Programmiersprachen möglich wäre.

Nur wenn das Thema klar auf ein bestimmtes Framework oder eine bestimmte Sprache ausgelegt ist,  ist eine technologisch spezifische Konzeption angemessen.

\subsection{Grafiken}
Beim Layout gibt es viele Details, die nicht kritisch sind. Zwei Punkte sind jedoch wichtig. Die in Grafiken verwendete Schriftart sollte identisch oder ähnlich zur Textschrift sein. Außerdem sollte die Schriftgröße sich am Fließtext orientieren. In der Praxis lässt sich die passende Größe meist nur durch Ausprobieren finden. In nahezu allen professionellen Veröffentlichungen ist dies Standard – daran sollte man sich frühzeitig gewöhnen, da es die Lesbarkeit und den Gesamteindruck der Arbeit deutlich verbessert. Ein Beispiel kann man finden in \Cref{subsec:grafiken}. 

\section{Genereller Aufbau}

Eine klassische Bachelorarbeit ist typischerweise wie folgt aufgebaut und kann als grundlegende Orientierung dienen.

\begin{itemize}
    \item Einleitung
    \item Grundlagen
    \item Konzeption
    \item Implementierung
    \item Methodik
    \item Ergebnisse
    \item Diskussion
    \item Zusammenfassung
\end{itemize}

Nicht in jeder Arbeit werden alle genannten Kapitel benötigt. Ein Methodik-Kapitel ist beispielsweise vor allem dann sinnvoll, wenn Experimente oder empirische Untersuchungen durchgeführt werden. Die dargestellte Struktur dient primär als Orientierungshilfe und muss nicht eins zu eins übernommen werden. Generell gilt es folgende grundlegende Punkte zu beachten. 

\paragraph{Klare Trennung der Kapitel}
Konzeption und Implementierung sind klar voneinander zu trennen. Ebenso müssen Ergebnisse und Diskussion separat behandelt werden: Während die Ergebnisse objektiv dargestellt werden, erfolgt die Interpretation erst in der Diskussion. Die Methodik beschreibt ausschließlich die Vorgehensweise und wird nicht mit anderen Inhalten vermischt.

\paragraph{Architektur- und Technologieentscheidungen begründen}
Die Auswahl von Frameworks und Technologien (z.\,B. Next.js oder gRPC vs.\ REST) sollte stets anhand sachlicher Kriterien erfolgen, etwa Community-Größe, Stabilität, Hosting-Möglichkeiten, Performance oder dem konkreten Use Case. Persönliche Präferenzen („damit kenne ich mich gut aus“) sollten dabei nicht im Vordergrund stehen. Es ist wichtig aufzuzeigen, welche Alternativen es gegeben hätte und warum letztlich eine bestimmte Lösung gewählt wurde. So wird nachvollziehbar, dass die Entscheidungen bewusst getroffen wurden und nicht zufällig entstanden sind.

\paragraph{Anforderungsanalyse}
Die Anforderungsanalyse ist ein Teil der Konzeption und sollte in jeder Abschlussarbeit vorhanden sein. Sie beschreibt systematisch, was das System leisten soll und unter welchen Rahmenbedingungen es betrieben wird. Dabei wird häufig zwischen Akteuren, Use Cases, User Stories sowie funktionalen Anforderungen (FA) und nicht-funktionalen Anforderungen (NFA) unterschieden.

\paragraph{Reflexion}
Die Diskussion sollte einen Reflexionsteil enthalten. In diesem werden die Limitationen der eigenen Arbeit sowie mögliche Bias offen benannt und kritisch eingeordnet. Ziel ist es zu zeigen, dass die Grenzen der Ergebnisse verstanden werden und realistisch eingeschätzt werden können.

\section{Abgabe}
Der erarbeitete Source Code muss zusammen mit der Arbeit abgegeben werden. Dieser ist allerdings meist zu groß, um ihn per E-Mail zu verschicken. Die empfohlene Variante ist, die Dateien in der BHT-Cloud\footnote{\url{https://cloud.bht-berlin.de/index.php/login}} hochzuladen und den entsprechenden Link in die Abgabe-E-Mail einzufügen. 

Bei komplexen Softwaresystemen kann eine kurze Demo oder ein Screencast hilfreich sein. Für Gutachter ist es oft schwierig, allein anhand der schriftlichen Beschreibung einen vollständigen Eindruck des entwickelten Systems zu gewinnen. Dies ist jedoch keine Pflicht. Falls vor der Abgabe nicht ausreichend Zeit vorhanden ist, kann ein Screencast auch noch etwa eine Woche nach der Abgabe nachgereicht werden. In der Praxis erfolgt die Begutachtung häufig erst nach mehreren Wochen, sodass hierfür meist ein zeitliches Fenster besteht.

\section{Erfolgreiche Bachelorarbeiten}
Hier sind Links von erfolgreichen Bachelorarbeiten zu finden, welche als Inspiration dienen können. 



TODO: Link einfügen






