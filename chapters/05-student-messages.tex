\chapter{Erfahrungen von anderen Studenten}
Dieses Kapitel dient dazu, allgemeine Erkenntnisse und gesammelte Erfahrungen festzuhalten, die sich im Verlauf verschiedener Abschlussarbeiten ergeben haben und für andere Studierende hilfreich sein können.

\subsection*{Julian}

Für Diagramme würde ich beim nächsten Mal direkt Inkscape verwenden, da draw.io teilweise Probleme mit der Vektorisierung von Schriftarten hat. Es gibt zwar Workarounds (Export als PDF, Import in Inkscape, dort vektorisieren und anschließend wieder in draw.io einbinden), aber das hat nicht gut funktioniert. Inkscape ist für SVG-Grafiken besser geeignet und bietet insgesamt professionellere Möglichkeiten.

Siamak in seiner Dissertation \cite{Haschemi2015} jedes Kapitel mit einer kurzen Zusammenfassung des vorherigen Kapitels sowie einer kurzen Vorschau auf das aktuelle Kapitel begonnen. Das hilft Lesenden sehr bei der Orientierung und würde ich in zukünftigen Arbeiten ebenfalls übernehmen.


